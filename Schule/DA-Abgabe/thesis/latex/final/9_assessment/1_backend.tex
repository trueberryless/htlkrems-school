\section{Messaging-System und Backend}
\label{chp:assessment_backend}

Es ist für die Implementation des Backend-Systems von Siemens die klare Anforderung gekommen, dass der Prototyp dieser Arbeit eine starke Entkoppelung zur Siemens GNA Software haben soll. Dadurch ist die Frage entstanden, welche Methode der Message-Propagation in verteilten Systeme optimal und vorteilhaft für diesen Fall wäre. Am Anfang dieser Arbeit ist die direkte Kommunikation, der Austausch über eine gemeinsame Datenbank und die Kommunikation über ein Messaging-System miteinander verglichen worden. Daraus ist hervorgegangen, dass Messaging-Systeme die größte Entkopplung bieten, welche sie zur geeignetsten Lösung machen. Anschließend sind RabbitMQ und Apache Kafka, welche populäre Implementierungen von Messaging-Systemen sind, für den Einsatz in dem Anwendungsfall dieser Arbeit bewertet worden. Apache Kafka hat sich als Sieger herausgestellt, denn für die Verarbeitung der Fehlerdaten wird eine hohe Durchsatzrate an Nachrichten beim Übermitteln benötigt, welche RabbitMQ nicht liefern kann. 

Diese Erkenntnisse haben sich für die Implementierung des Backend-Systems als sehr wertvoll und brauchbar herausgestellt. Es ist durch die Verwendung des Kafka-Systems eine starke Entkopplung der Systeme erreicht worden, wobei keine Beeinträchtigungen bezüglich der Übertragungsgeschwindigkeit bemerkbar geworden sind. Das Java Spring-Framework hat die Entwicklung des Prototyps vereinfacht und somit ist eine wartbare und übersichtliche Applikation entstanden. Während des Entstehens des Prototyps haben sich die Erweiterungen des Spring-Frameworks, welche die Integration von fremden Systemen ermöglichen, als besonders vorteilhaft herausgestellt. Beispielsweise ist die Integration von Apache Kafka in die Applikation durch das Erweiterungspaket \emph{Spring for Apache Kafka} mit nur niedrigen Aufwand vollständig gelungen.

Die einzige Komplikation, welche bei der Implementierung aufgetreten ist, war die Verwendung einer relationalen Datenbank für stark vernetzte Daten. Die PostgreSQL Datenbank ist nicht bei herkömmlicher Verwendung den Anforderungen nachgekommen, insbesondere bezüglich der Einfüge-Performance. Dadurch ist in der Entwicklung des Prototyps auf eine besondere Technik gesetzt worden, welche gegen die Grundsätze der relationalen Datenbanken verstößt. Jedoch kann damit trotzdem die benötigte Einfüge-Performance und die geeignete Abfragegeschwindigkeit erreicht werden. Im Nachhinein wäre die Verwendung einer anderen Art von Datenbank, wie zum Beispiel einer Graph-Datenbank von großem Vorteil gegenüber PostgreSQL gewesen. Es muss angemerkt werden, dass die Wahl einer anderen Datenbank im Kontext unserer Kooperation mit Siemens aber leider nicht möglich gewesen ist.