\section{Visualisierung des Netzmodells}
\label{chp:assessment_frontend}

Zu Beginn dieser Arbeit sind verschiedene Methoden, welche die Einfachheit einer Visualisierung sicherstellen, vorgestellt worden. Hierzu zählen die Intuitivität, Interaktivität und Performance einer Applikation sowie die Reduktion, das Clustering und die Verzerrung der Daten. Einige dieser Methoden sind auch im Prototypen implementiert worden, um deren Korrektheit zu beweisen und einen Vergleich aufzustellen. Schlussendlich stellt sich natürlich die Frage, ob die verwendeten Frameworks und Bibliotheken im Frontend eine gute Entscheidung gewesen sind oder ob für diese Art von Anforderungen eine andere Architektur zu bevorzugen ist.

\subsection{Intuitivität, Interaktivität und Performance}

Unsere Applikation ist intuitiv aufgebaut. Die Filtermöglichkeiten der Tabelle sind ziemlich selbsterklärend, sodass Benutzer nicht lange darüber nachdenken müssen, was diese Checkboxen eigentlich machen. Auch die Interaktion mit der Tabelle in Bezug auf Sortierung und Pagination ist äußerst intuitiv, weil diese Darstellungen (Pfeile und Button) heutzutage bereits zum Standard des Designs geworden sind.

Doch nicht nur die Tabelle ist intuitiv und interaktiv. Auch der Graph verfügt über derartige Eigenschaften. Aufgrund der Funktionalität, dass die Knoten und Kanten herumgeschoben werden können, hinein- und auch hinausgezoomt werden kann, wird der Benutzer dazu verleitet, das Netzwerk zu erkunden. Diese Interaktivität fördert insbesonders das explorative Verhalten des Stromnetzmodells. Weiter interaktive Effekte, wie zum Beispiel das bewegliche \emph{Fisheye} und die kartesische Verzerrung könnten auf den Graphen angewandt werden. Des Weiteren ist die Darstellung des Dashboards einfach zu verstehen und bedienen. Der Benutzer benötigt jedoch eventuell gewisse Vorkenntnisse bezüglich des Aufbaus des Netzmodells, um genau zu verstehen, was die Attribute der Fehlerdaten und Modellobjekte in Bezug zum Stromnetzwerk aussagen.

Zum Thema Performance lässt sich sagen, dass aufgrund der Verwendung von GraphQL sehr wenige HTTP-Abfragen getätigt werden müssen. Immerhin können pro Seite fast alle Daten mit einem einzigen Request zum Client gesendet werden. Das einzige argumentative Manko ist die Ladezeit des Graphen, welche jedoch nicht am Backend liegt. Denn das Backend verarbeitet Anfragen der Modellobjekte in Bezug auf die Tiefe linear. Der Algorithmus zum Erstellen der Knoten- und Kantenelemente auf Seiten des Frontends ist jedoch aufgrund der vielen möglichen Verbindungen zwischen den Knoten beeinträchtigt.

\subsection{Vereinfachung von komplexen Daten für die Visualisierung}

Der Graph der Applikation reduziert einerseits die Daten auf eine kleiner Menge, sodass die Benutzer diesen Teil des Netzmodells effizienter verstehen können. Dadurch können die Ingenieure von Siemens fundierte Entscheidungen treffen und zum Beispiel einen Beauftragten zu dem realen Objekt schicken, damit dieser das Problem schnellsmöglich beheben kann. Diese Reduktion funktioniert aufgrund der Einschränkung mittels Startobjekt und gewisser Tiefe. Das Stromnetzwerk erlaubt aufgrund seiner inhärenten Muster jedoch auch eine Reduktion der Daten ohne \wordindoublequotes{Eingangspunkt}. Denn wie bereits im empirischen Teil erwähnt worden ist, teilt sich das Netzwerk in eine aktive und eine passive Gruppe, wobei aktive Elemente Generatoren, Schalter, Sicherungen, Umspannwerke, und so weiter sein. Die andere Gruppe dieser Bipartition (Sektion \ref{chp:bipartition}) besteht hauptsächlich aus verschiedensten Leitern. Somit muss bei der Reduktion darauf geachtet werden, dass Elemente von beiden Gruppen unter den gefilterten Knoten vorkommen, damit Verbindungen zwischen diesen existieren können.

Eine andere Möglichkeit ist das Teilen mittels Clustern. Falls die Daten über kategoriale Eigenschaften verfügen, können diese genutzt werden, um nur alle Knoten einer Kategorie zu laden. Optimalerweise treten die verschiedenen Kategorien gleichmäßig verteilt auf, sodass zum Beispiel nicht alle Datensätze die gleiche Kategorie haben. Dies wäre keine Reduktion der Daten. Cluster können jedoch auch mit anderen Merkmalen gebildet werden. Beispielsweise kann mittels Adresse der Graph auf eine bestimmte Region eingeschränkt werden. Oder ein künstlicher Algorithmus untersucht die Daten und teilt diese in Cluster.

Neben Reduktion sind auch farbliche Differenzen auf allen Seiten verwendet worden, um Inhalte gegenüber anderen hervorzuheben. Das Dashboard lenkt somit die Aufmerksamkeit auf die verschiedenen aussagekräftigen Kreissegmente, welche die Verteilung von Klasse, Kategorie, Schweregrad und Spannungslevel der Ergebnisse des Netzmodells betonen. Die Tabelle verwendet Farben sowohl für klassische UI-Elemente, als auch den Ladebalken, und im Graphen wird das aktuelle Objekt (Startobjekt) farblich markiert. Allgemein ist die Umsetzung des \emph{Corporate Designs} äußerst positiv bei unseren Korrespondenten bei Siemens aufgefallen.

Die Gründe für einige Design-Entscheidungen, wie zum Beispiel Donatdiagramme im Dashboard oder die Filtermöglichkeiten bei der Tabelle, sind zum Teil die Einhaltung der Bitten von Siemens, da unser Korrespondent die optimale Effizienz dieser Features direkt bestätigen konnte, und in manchen Fällen Vorlieben meiner Wenigkeit. Niemand hat uns vorgeschrieben, wie der Graph auszusehen hat, an welcher Stelle gefiltert werden soll und welche Art von Darstellungen Siemens haben will. Einige dieser Entscheidungen sind aus reinem Bauchgefühl heraus entstanden -- teilweise bereits zu Zeiten, wo das Mockup des Prototypen erstellt worden ist. Im Großen und Ganzen lässt sich meiner Meinung nach sagen, dass diese persönlichen Präferenzen zu einem runden Gesamtprodukt geführt haben, welches hoffentlich in Zukunft in der Routine der Ingenieure von Siemens eingesetzt werden kann.

\subsection{Wahl der Frameworks und Bibliotheken}

Die Entscheidung \emph{Angular} als Framework für das Frontend zu verwenden kann nicht bereut werden. Das Framework hat -- wie bereits erwähnt -- viele Features bereits vorkonfiguriert, gute Dokumentation, ein umfangreiches Ökosystem und es ist aus diesem Grund wirklich einfach mit diesem Framework schnell Applikationen aufzubauen. Außerdem legt diese einen potentiellen Grundstein für die zukünftige Webentwicklung von Siemens, da das Unternehmen eventuell mit dieser Technologie weiterentwickeln möchte.

In Anbetracht auf die Bibliotheken, welche beim Frontend verwendet worden sind, kann ich nur sagen, dass das Arbeiten mit \emph{Cytoscape} eine recht angenehme Erfahrung war. Man merkt beim Verwenden dieser, dass sich die Entwickler dieser Bibliothek von Grund auf Gedanken über die Anwendung der Bibliothek gemacht haben. Sie bietet verschiedenste Methoden an, um gleiche Aktionen tätigen zu können, je nachdem, was dem Verwender am besten gefällt. Dass die gesamte Dokumentation eine sehr lange Webseite ist, hat mir aufgrund der dadurch verloren gegangenen Orientierung nicht gut gefallen, jedoch hat man keine überdurchschnittlichen Ladezeiten bemerken können. Dennoch empfehle ich für moderne Dokumentationen das relativ neu erschienene Starlight-Framework -- eine Entwicklung des Astro Teams --, welches aufgrund des Markdown-Supports den Aufwand der Erstellung einer Dokumentation erheblich reduziert.

\emph{ng2-charts} ist meiner Meinung nach die beste Bibliothek für die Darstellung von Diagrammen, wenn Angular als Basisframework verwendet wird. Die Bibliothek ist einfach zu integrieren und hat eine moderne, jedoch minimalistische Dokumentation -- das genaue Gegenteil von Cytoscape. Nichtsdestotrotz würde ich eine detaillierte Dokumentation einer gut aussehenden bevorzugen. Erwähnenswert ist auf jeden Fall die standardmäßige Interaktivität der zur Verfügung gestellten Diagramme der Bibliothek. Diese bietet dem Benutzer nämlich automatisch Möglichkeiten, mit den Diagrammen auf informative Weise zu kommunizieren.