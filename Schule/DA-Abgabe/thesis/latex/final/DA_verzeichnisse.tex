%
% Literatur-, Abbildungs-, Tabellen- und Abkuerzungsverzeichnis
%


% Literaturverzeichnis ===>>>> siehe DA.bib
\nocite{*} %alle Einträge aus DA.bib werden übernommen, auch diese, die nicht im Text vorkommen
\bibliographystyle{babunsrt} %sortiere Verzeichnis nach Reihenfolge im Text
\bibliography{DA} %erstelle Verzeichnis

% erstellt ein File mit der Liste aller Bilder (Endung: *.lof)
\listoffigures

% erstellt ein File mit der Liste aller Tabellen (Endung: *.lot)
\listoftables

\renewcommand\lstlistlistingname{Quellcodeverzeichnis} 
\lstlistoflistings 

\chapter{Abk\"urzungsverzeichnis}

\vspace{3mm}

\textbf{AMQP} Advanced Message Queuing Protocol -- Ein offenes Netzwerkprotokoll zum Verwaltung von Nachrichten zwischen Anwendungen.

\vspace{5mm}

\textbf{API} Application Programming Interface -- Eine Schnittstelle, die es Anwendungen ermöglicht, miteinander zu kommunizieren und auf bestimmte Funktionen oder Dienste zuzugreifen.

\vspace{5mm}

\textbf{DNS} Domain Name System -- Ein System zur Umwandlung von Domainnamen in IP-Adressen, das für die Navigation im Internet verwendet wird.

\vspace{5mm}

\textbf{HTTP} Hypertext Transfer Protocol -- Ein Protokoll zur Übertragung von Daten über das Internet, das zur Übertragung von Webseiten, Bildern, Videos und anderen Ressourcen verwendet wird.

\vspace{5mm}

\textbf{REST} Representational State Transfer -- Ein Prinzip beziehungsweise Regeln, welche das Internet konzeptionell als eine Sammlung von zustandlosen Ressourcen ansieht und diese als Repräsentationen über Endpunkte zur Verfügung stellt.

\vspace{5mm}

\textbf{SOA} Service-Oriented Architecture -- Ein Architekturstil, der die Erstellung von Anwendungen aus unabhängigen, wiederverwendbaren Diensten betont, die über ein Netzwerk miteinander kommunizieren.

\vspace{5mm}

\textbf{STOMP} Streaming Text Oriented Messaging Protocol -- Ein Protokoll zur Nachrichtenübertragung über Netzwerke, das auf der Idee des Nachrichten-Brokers basiert.

\vspace{5mm}

\textbf{UI} User Interface -- Die Schnittstelle, über die ein Benutzer mit einem Computer oder einer Software interagiert.

\vspace{5mm}

\textbf{UX} User Experience -- Die Gesamtheit der Eindrücke und Emotionen eines Benutzers bei der Interaktion mit einem Produkt oder einer Dienstleistung.

\vspace{5mm}

\textbf{XMPP} Extensible Messaging and Presence Protocol -- Ein offenes Protokoll zur Nachrichtenübertragung und Anwesenheitsinformationen in Echtzeit. Es wird oft für Instant Messaging verwendet.
