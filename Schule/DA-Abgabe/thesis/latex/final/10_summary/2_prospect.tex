\section{Ausblick}

In Anbetracht der erzielten Fortschritte und Erkenntnisse bietet dieser Abschnitt einen Ausblick auf potenzielle zukünftige Entwicklungen und Implementierungsmöglichkeiten, um das untersuchte Thema weiter zu vertiefen und neue Perspektiven zu eröffnen.

Im Backend-System sind alle wesentlichen Ziele erreicht worden, aber es gibt darüber hinaus noch weitere Features, welche man in das System integrieren könnte. Einerseits besteht die Möglichkeit, noch weitere Filter in der GraphQL-API bezüglich der Tabellenansicht zu implementieren. Andererseits könnte man ein Benachrichtigungssystem in das Backend einbauen, welches bei Änderung der Daten in der Datenbank, eine Aktualisierung der Datenvisualisierung im Frontend auslöst.  

Aufseiten des Frontends gibt es vor allem bei dem Graphen noch wunderbare Ideen und Erweiterungen. Das bereits im Rechercheteil erwähnte Konzept des Clusterings ist nämlich mittels Bibliothek Cytoscape bereits möglich und könnte für Siemens eine äußerst nützliche und hilfreiche Anwendung finden. Die Struktur der Daten erlaubt diese Implementierung ebenfalls, da sie eine Art Baumstruktur mit verschachtelten Containern aufweist.

Außerdem könnte der interessante Fisheye-Effekt das Anzeigen von mehr Objekten im Graphen ermöglichen, da dem Benutzer die Fähigkeit des intuitiven Heranzoomens gegeben wird. Eine derartige Arbeit könnte die Frage der Verbesserung der Effizienz bei Analyse- und Entscheidungsprozessen wortwörtlich unter die Lupe nehmen.

Abschließend lässt sich noch die Idee in den Raum werfen, zusätzliche relevante Informationen kompakt auf dem Dashboard anzuzeigen. Beispielsweise könnte eine kleine Version des Graphen mit Startelement eines schwerwiegenden Fehlers dort Platz finden.